\section{Einleitung}
\subsection{Problemstellung}
Digitale Authentifizierungstechnologien bewegen sich in einem strategischen Spannungsfeld zwischen Sicherheit und Usability. Diese Dualität stellt Unternehmen wie Privatpersonen vor die Herausforderung, sensible Daten und Zugriffe nachhaltig zu sichern. Die klassische Ein-Faktor-Authentifizierung (1FA) kann dieses Problem nicht mehr lösen, da der Faktor „Wissen“ (meist ein Passwort) durch menschliche Schwächen wie Vergesslichkeit oder Bequemlichkeit anfällig ist und dessen Manipulierbarkeit in der Literatur umfangreich dokumentiert wurde (Bonneau et al., 2015, S. 81).
Gleichzeitig verzeichnen europäische Organisationen einen massiven Anstieg von identitätsbezogenen Angriffen (European Union Agency for Cybersecurity, 2024, S. 27). Angriffsvektoren wie Phishing, Identitätsbetrug und Malware bedrohen dabei private und korporative Akteure gleichermaßen. In diesem Kontext stellt die alleinige Nutzung von Passwörtern ein inakzeptables Risiko dar (Das et al., 2020, S. 5446).
Multifaktor-Authentifizierung (MFA) schließt diese Sicherheitslücke, indem sie zusätzliche, (voneinander) unabhängige Faktoren in die Identitätsfeststellung integriert (O’Gorman, 2003, S. 2023). Insbesondere in sensiblen Sektoren wie dem Gesundheitswesen, dem Bankensektor oder der öffentlichen Verwaltung hätte eine Kompromittierung verheerende Folgen, weshalb verbindliche Standards und Guidelines hier dringend erforderlich sind (EBA, o. J., S. 12).
Die vorliegende Arbeit untersucht, welche Faktoren die effektive Nutzung von MFA beeinflussen. Während Aspekte der Usability und Akzeptanz als Rahmenbedingungen betrachtet werden, liegt der primäre Fokus auf IT-Governance als steuerndem Element. Obwohl Governance üblicherweise im Unternehmenskontext verortet wird, gewinnt sie durch die wachsende Anzahl an IT-Regularien (NIS-2 , DORA, FIDA, CRA) zunehmend auch für Privatpersonen an Relevanz, da berufliche Sicherheitsvorgaben und private Nutzungsgewohnheiten zunehmend verschwimmen.
\subsection{Zielsetzung und wissenschaftliche Fragestellung}
Multifaktor-Authentifizierung (MFA) stellt weit mehr dar als eine rein technische Standardlösung für das Identitäts- und Zugangsmanagement. Sie bewegt sich in einem komplexen Anforderungsfeld: Die Technologie muss nicht nur kryptografisch sicher und resistent gegen Täuschung sein, sondern gleichzeitig externe Compliance-Anforderungen erfüllen und durch eine hohe Benutzerfreundlichkeit (Usability) Risiken minimieren, die durch menschliches Fehlverhalten entstehen (Kim et al., 2011, S. 187).
In diesem Kontext nimmt IT-Governance eine Schlüsselrolle ein. Durch Richtlinien und strategische Vorgaben können Unternehmen Sicherheitsaspekte mit den Interessen der Stakeholder in Einklang bringen (Wibowo & Ramli, 2022, S. 2). Auf Basis der ermittelten Authenticator Assurance Levels (AAL) werden Technologien und Prozesse definiert, um diese Ziele zu erreichen. Damit verortet sich das Thema nachvollziehbar im Bereich des strategischen IT-Infrastruktur-Managements.
Die zentrale Forschungsfrage der vorliegenden Arbeit lautet:
Inwiefern wirkt IT-Governance als Treiber für die Akzeptanz und Nutzung von Multifaktor-Authentifizierung im Vergleich zwischen beruflichen und privaten Kontexten?
Zur detaillierten Beantwortung werden folgende Unterfragen untersucht, um die regulatorischen und psychologischen Dimensionen herauszuarbeiten:
Welche regulatorischen und organisationalen Mechanismen erzwingen die Nutzung im beruflichen Umfeld?
Welche Barrieren verhindern die freiwillige Nutzung im privaten Umfeld?
Methodisch erfolgt die Bearbeitung durch eine systematische Literaturrecherche (Literature Review). Dabei werden relevante wissenschaftliche Publikationen, Fachartikel und Standards (z. B. NIST, ISO) analysiert, um sowohl die technischen Aspekte von MFA als auch die organisatorischen Governance-Faktoren zu beleuchten. Ziel ist es, die Nutzungsmuster im beruflichen und privaten Kontext gegenüberzustellen, Unterschiede herauszuarbeiten sowie Erfolgsfaktoren und Hindernisse zu identifizieren.
Die Relevanz dieser Seminararbeit ergibt sich insbesondere für (angehende) IT-Manager:innen sowie Verantwortliche im Bereich Identity- und Access-Management (IAM), um Entscheidungen mit stärkerem Nutzer:innen-Fokus zu treffen und Organisationen auf die steigende Dichte an regulatorischen Vorgaben vorzubereiten.
