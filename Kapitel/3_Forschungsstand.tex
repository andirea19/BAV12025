/section{Forschungsstand}
3.1	Herausforderungen
3.1.1	MFA als technische Herausforderung
Die technische Implementierung von Multifaktor-Authentifizierung konfrontiert IT-Verantwortliche mit komplexen Herausforderungen, die weit über die bloße Auswahl eines zweiten Faktors hinausgehen. Der aktuelle Forschungsstand beschreibt ein Spannungsfeld, das von der Integration in bestehende Legacy-Architekturen über den Lifecycle der Token bis hin zur Absicherung der Recovery-Prozesse reicht.
Grundlegende Publikationen zu modernen Authentifizierungsmechanismen beschreiben diese Architekturen und definieren prinzipielle Anwendungsszenarien für unterschiedliche Sicherheitsniveaus (Ometov et al., 2018, S. 4). Ein besonderer Fokus liegt dabei auf der Ablösung unsicherer Verfahren: Konkrete Technologien wie FIDO-basierte Lösungen werden in der Literatur als vielversprechende, phishing-resistente Alternative zu klassischen OTP-Verfahren bewertet (Lyastani et al., 2020, S. 269). Ergänzend dazu widmen sich Reifegradanalysen der Frage, wie weit verbreitet diese Techniken in verschiedenen Branchen bereits sind und welche technischen Barrieren, wie fehlende Schnittstellen oder Legacy-Systeme, die Verwendung erschweren (Savaş et al., 2022, S. 2–4).
Spezifische Branchenanalysen zeigen, dass insbesondere im stark regulierten Finanzsektor hohe Hürden bestehen. Studien analysieren hier Implementierungsstrategien, die den Spagat zwischen strengen Compliance-Vorgaben und technischer Machbarkeit in heterogenen Banking-Umgebungen meistern müssen (Aburbeian & Fernández-Veiga, 2024, S. 178–180). Ein weiteres zentrales Forschungsfeld ist die geräteübergreifende Identitätsverwaltung. Arbeiten in diesem Bereich thematisieren, wie MFA konsistent und sicher in Bring-Your-Own-Device (BYOD)-Szenarien betrieben werden kann, ohne die Hoheit über die Unternehmensdaten zu verlieren (Hilbig et al., 2024, S. 74).
Eine häufig identifizierter kritischer Schwachpunkt sind die Account-Recovery-Mechanismen. Robuste und zugleich benutzerfreundliche Wiederherstellungsverfahren (zum Beispiel bei Token-Verlust) gelten als technisch schwer umzusetzen, da sie oft neue Angriffsvektoren für Social Engineering eröffnen (Keil & Zugenmaier, 2024, S. 73). Parallel dazu untersuchen aktuelle Arbeiten die Cloud-Integration von Identitätsdiensten (Identity-as-a-Service, IDaaS). Hierbei stehen Infrastruktur-, Datenschutz- und Verfügbarkeitsfragen im Vordergrund, die für die Resilienz cloudbasierter MFA-Angebote entscheidend sind (Otta et al., 2023, S. 4).
Abschließend widmet sich ein Teil der Forschung der Evolution der Faktoren selbst. Arbeiten zur Biometrie beleuchten deren Rolle als komfortabler, aber datenschutzrechtlich sensibler Bestandteil von MFA-Szenarien (Wolf et al., 2019, S. 151; Trewin et al., 2012, S. 159). Darüber hinaus untersuchen Beiträge auch experimentelle Authentifikationsformen, etwa auf Basis von Schallwellen, Bewegungs- oder Geräuschmustern. Diese liefern Hinweise auf alternative Faktoren für spezifische Nischenanwendungen, befinden sich jedoch meist noch im Stadium experimenteller Prototypen (Karapanos et al., 2015, S. 483).
3.1.2	MFA als Herausforderung an Nutzer:innen
Zusätzlich zu den vorgestellten, technischen Fragestellungen konzentrieren sich einige Publikationen auf die Perspektive der Nutzenden. Aspekte wie Akzeptanz und Benutzerfreundlichkeit entscheiden mit über die tatsächliche Sicherheit von MFA-Systemen. Studien zur Awareness von Nutzenden zeigen, dass mangelndes Verständnis für Funktionsweise und Grund der Anwendung zu seltener Nutzung oder Umgehungsverhalten führt (Claudia Ziegler Acemyan et al., 2018). Weitere Untersuchungen und Feldstudien beleuchten die Verbreitung und Wahrnehmung von FIDO-Technologien in konkreten Kontexten, beispielsweise in Deutschland, und liefern Hinweise auf Blockaden und Treiber für die Verwendung (Keil & Zugenmaier, 2024, S. 76f). 
Historische Betrachtungen des Übergangs zu verpflichtender 2FA in bestimmten Sektoren zeigen, wie regulatorische Vorgaben und Compliance-Anforderungen Adoptionstreiber sein können (Wibowo & Ramli, 2022, S. 2). Zugleich können erweiterte beziehungsweise neue Vorgaben aber auch Usability-Probleme  Supportaufwände erhöhen (Thanasis Petsas et al., 2015). Usability-Analysen spezifisch zu Hardware-Security-Keys und FIDO-Keys untersuchen Aspekte wie Setup-Komplexität, Fehlerquoten und subjektive Zufriedenheit und kommen zu dem Schluss, dass physische Keys zwar hohe Sicherheit liefern, aber in Teilen der Nutzerpopulation auf Akzeptanzhürden stoßen (Stéphane Ciolino et al., 2019; J. K. Reynolds et al., 2018).
Weitere empirische Arbeiten adressieren soziale Determinanten und Kontextfaktoren der Akzeptanz: Einflussgrößen beziehungsweise demographische Variablen wie Geschlecht oder Alter wurden in mehreren Studien untersucht (Gratian et al., 2018, S. 349f), wobei heterogene Ergebnisse darauf hindeuten, dass Akzeptanz individuell erklärbar ist und kontextuelle Faktoren (zum Beispiel Anwendungsszenario, Schulung, Vorerfahrungen) eine große Rolle spielen (Mohd Anwar et al., 2017). Feldstudien an Hochschulen und in Organisationen liefern zusätzliche Einsichten in Adoptionserfahrungen, typische Barrieren sowie praktikable Maßnahmen zur Steigerung der Nutzerakzeptanz (Jessica Colnago et al., 2018; Jonathan Dutson et al., 2019). Mehrere Analysen, warum Nutzer:innen MFA nicht akzeptieren oder umgehen, fassen häufig Gründe wie wahrgenommene Komplexität, fehlende Nutzenwahrnehmung und schlechte Recovery-Erfahrungen zusammen (Sarah Pearman et al., 2019). Gleichzeitig wurden MFA-Systeme in Studien als ein Zeichen von Sicherheit erlebt. So beschreiben beispielsweise Krol et al., (2015, S. 2), dass mehrere Teilnehmende ihrer Studie die zusätzliche Sicherheit bei Banking-Anwendungen schätzen.
3.1.3	MFA als IT-Governance  Herausforderung
Die vorgestellte Literatur verdeutlicht, dass die Einführung und Verwendung von Authentifizierungsmechanismen auch eine Governance-Aufgabe ist. Governance umfasst in diesem Kontext die strategische Steuerung, Regulierung und Überwachung von Prozessen innerhalb von Organisationen (Savaş et al., 2022).
Studien zu Governance Modellen in Bezug auf Sicherheit zeigen, dass Reifegrad, Verantwortlichkeiten und Kommunikation für die nachhaltige Wirksamkeit von MFA-Systemen wichtig sind (Endro Joko Wibowo et al., 2022, S. 4–6). Unternehmen mit klar definierten Policies und kontinuierlichem Monitoring erreichen nachweislich höhere Sicherheits- und Akzeptanzwerte (Iryna Krykavska, 2023, S. 7). 
IT-Governance Frameworks wie die ISO 27001 oder die NIST Digital Identity Guidelines (Temoshok et al., 2025) machen diese Steuerung anwendbar, indem sie notwendige Authenticator-Assurance-Levels (AAL) und Anforderungen an die Technologien festlegen. Auf regulatorischer Ebene verpflichten PSD2 und eIDAS Organisationen zur Nutzung starker Authentifizierung, was Governance-Strukturen erforderlich macht, die technische und rechtliche Vorgaben integrieren (Govindraj C.V., 2024; S. 4).
Im öffentlichen Sektor rückt damit die Frage nach digitaler Souveränität und Vertrauensinfrastruktur in den Vordergrund: Studien zu E-Government-Governance betonen, dass nachhaltige Sicherheitsstrategien nur durch ein Zusammenspiel von technischen Standards, institutioneller Verantwortung und nutzer:innenzentrierte Politik erreicht werden können (A. Ariyadi & M. Akbar, 2025, S. 2).
In Unternehmenskontexten treten zusätzliche organisatorische Herausforderungen auf: Vendor Lock-in, Cloud-Abhängigkeiten, Helpdesk-Aufwand bei Tokenverlust oder vergessenem Passwort und organisatorische Integrationsprobleme in bestehende IAM-Strukturen ((Das et al., o. J., S. 4–6); Temoshok et al., 2025, §5).
Eine konkrete Herausforderung sind auch informelle Systeme zur Umgehung, die sogenannte Schatten/Shadow-IT, die häufig als Antwort auf Governance Vorschriften entsteht (Fagan & Khan, 2016, S. 65–67). 
Aktuelle regulatorische Entwicklungen wie die NIS2-Richtlinie verstärken außerdem die Anforderungen an die organisationalen Nachweispflichten und das Management kritischer Dienste immer weiter.  entsprechende Analysen zu NIS2 sind für die Ausgestaltung von MFA-Governance empfehlenswert (Barton & Lewis, 2024). Ebenso sollten neuere Arbeiten zur Lieferketten- und Vendor-Governance in MFA-Kontexten (zum Beispiel Lassak et al., 2024) berücksichtigt werden, da Third-Party-Provider und Cloud-Identitätsdienste zentrale Risiken und Abhängigkeiten mit sich bringen.
Aus Governance Sicht erfordert die erfolgreiche Implementierung von MFA sowohl technische Kompetenz, wie auch klare Zuständigkeiten und Awareness-Programme beziehungsweise Schulung (Hans de Bruijn et al., 2017, S. 2).
3.2	Wirkmechanismen der Nutzung von MFA
Die Analyse der Literatur zeigt, dass IT-Governance nicht isoliert im Unternehmenskontext wirkt, sondern als Katalysator für privates Sicherheitsverhalten fungiert. Konrekt lassen sich drei Wirkmechanismen identifizieren:
Erstens der Kompetenz-Transfer: Die größte Hürde für private MFA-Nutzung ist oft nicht der Widerwille, sondern die technische Unsicherheit. Strenge Governance-Richtlinien im Beruf zwingen Nutzer:innen, diese Hürde zu überwinden. Studien wie die von Colnago et al. (2018) bestätigen diesen Lerneffekt: Userinnen, die Usability von MFA-Token im „Safe Space“ des Unternehmens (mit Support-Netz) kennenlernen, trauen sich deren Anwendung auch privat zu. Governance wirkt hier somit sozusagen als unfreiwilliges Schulungsprogramm.
Zweitens die unterschiedliche Risikowahrnehmung (Awareness): Während private Nutzer:innen Risiken oft abstrakt einschätzen, wird Risikomanagement in Unternehmen konkretisiert. Richtlinien, die auf NIS-2 oder DSGVO basieren, kommunizieren intern wie extern explizit, warum was geschützt wird. Dieser Sensibilisierung schwappt über: Das Verständnis, dass Identitäten und Daten schützenswert sind, bleibt nach Feierabend bestehen (Spill-over). Governance verändert MFA somit von einer „lästigen Schikane“ zu einem „notwendigen Standard“. 
Drittens der persönliche Widerstand bzw. Reaktivität: Mehrere Quellen zeigten  auch eine negative Auswirkung von Governance. Ist die Governance zu starr und die Usability schlecht (zum Beispiel häufige Re-Authentifizierung ohne biometrische Optionen), entsteht „Security Fatigue“ (Ermüdung). Dies führt nicht zu privater Adoption, sondern zu einer Abwehrhaltung. Nutzer:innen vermeiden privat Sicherheitsmaßnahmen, um ihre „kognitive Ruhe“ zu beschützen.
3.3	IT-Governance als Treiber der Akzeptanz und Nutzung von Multifaktor-Authentifizierung
Die vorgestellten Quellen bestätigen also zusammenfassend das Zusammenspiel zwischen dem durch IT-Governance gesteuerten, beruflichen Umfeld, und dem von Usability und persönlicher Wahrnehmung geprägten privaten Kontext.
Studien zu Governance-Modellen im Bereich IT-Sicherheit zeigen, dass Reifegrad, klar definierte Verantwortlichkeiten und transparente Kommunikationsstrukturen entscheidend für die nachhaltige Wirksamkeit von MFA-Systemen sind (Endro Joko Wibowo et al., 2022, S. 4–6). Organisationen, die über klar formulierte Policies, standardisierte Prozesse und ein kontinuierliches Monitoring verfügen, erreichen nachweislich nicht nur höhere Sicherheitsstandards, sondern auch eine bessere Nutzer:innenakzeptanz.
IT-Governance-Frameworks wie die ISO 27001 oder die NIST Digital Identity Guidelines (Temoshok et al., 2025) stellen Instrumente bereit, mit denen Organisationen Sicherheitsmaßnahmen systematisch planen, steuern und evaluieren können. Sie legen unter anderem Authenticator-Assurance-Levels (AAL) fest, die die Anforderungen an die Stärke der Authentifizierung definieren, und geben technische Spezifikationen vor, die die Implementierung von MFA erleichtern. Diese Frameworks unterstützen Organisationen dabei, die Balance zwischen Sicherheit, Benutzerfreundlichkeit und regulatorischer Compliance zu halten.
Auf regulatorischer Ebene verpflichten Vorgaben wie PSD2 und eIDAS Unternehmen zur Nutzung starker Authentifizierung. Dies erfordert Governance-Strukturen, die technische, organisatorische und rechtliche Anforderungen integrieren, um sowohl interne Sicherheitsziele als auch externe Vorschriften zu erfüllen (Govindraj C.V., 2024; Iryna Krykavska, 2023).  Die Einhaltung solcher regulatorischen Anforderungen ist insbesondere in hochsensiblen Branchen wie Finanzwesen oder öffentlicher Verwaltung kritisch, da Verstöße erhebliche rechtliche Konsequenzen und Vertrauensverluste nach sich ziehen können.
Im öffentlichen Sektor gewinnt zudem die digitale Souveränität an Bedeutung: E-Government-Initiativen erfordern sichere und vertrauenswürdige Infrastrukturen, die den Schutz sensibler Bürgerdaten gewährleisten. Studien zu E-Government-Governance betonen, dass nachhaltige Sicherheitsstrategien nur durch ein Zusammenspiel von technischen Standards, institutioneller Verantwortung und nutzerzentrierter Politik erreicht werden können (A. Ariyadi & M. Akbar, 2025, S. 2).
Es gibt Hinweise auf einen sogenannte Spill-over-Effekt wie ihn Staines (1980) in einem Arbeitspsychologischen Kontext beschrieb (Staines, 1980, S. 120–121). Konkret insofern, dass positive Erfahrungen mit gut implementierter MFA im beruflichen Kontext (beispielsweise durch erfolgreiche IT-Governance) dazu führen können, dass Nutzer:innen MFA auch privat aktivieren. Nutzende, die im beruflichen Kontext gute Erfahrungen mit einer MFA-Lösung (wie Push-Benachrichtigungen) gemacht haben, neigen eher dazu, diese Technologien auch für private Accounts zu aktivieren (Jessica Colnago et al., 2018, S.).
Umgekehrt führen negative Erfahrungen oder mangelndes Verständnis von Risiken dazu, dass Nutzende MFA-Technologie privat ablehnen, weil sie glauben, ihre privaten Daten seien weniger wichtig. Wenn Nutzer:innen im beruflichen Kontext beispielsweise negative Erfahrungen machen, sinkt die Akzeptanz. Krol et al. (2015, S. 3) stellten fest, dass Kund:innen im privaten Banking sogar den Finanzdienstleister wechselten, weil sie umständliche MFA-Hardware vermeiden wollten. Nutzende, welche negative Konsequenzen erleben, raten anderen aktiv von der Nutzung ab und beeinflussen so ihr Umfeld mit (Das et al., 2020, S. 2).
/subsection{Beantwortung der Forschungsfrage}
Die vorliegende Arbeit untersucht das Spannungsfeld zwischen beruflicher Pflicht und privater Freiwilligkeit bei der Nutzung von Multifaktor-Authentifizierung. Ziel ist es, den Einfluss von IT-Governance auf dieses Nutzungsverhalten zu analysieren. Dabei soll folgende Forschungsfrage beantwortet werden: Inwiefern wirkt IT-Governance als Treiberin für die Akzeptanz und Nutzung von Multifaktor-Authentifizierung im Vergleich zwischen beruflichen und privaten Kontexten?
Die Untersuchung zeigt deutlich, dass IT-Governance über die Unternehmensgrenzen hinaus wirksam ist. Sie tritt als treibender Faktor auf, um die Brücke zwischen notwendiger Sicherheit und gewünschter Bequemlichkeit (Usability) zu schlagen. Während im privaten Kontext oft die Bequemlichkeit vorrangig ist und MFA häufig als Störfaktor wahrgenommen wird, etabliert Governance im beruflichen Kontext – getrieben durch Regulatorien wie NIS-2 und DORA – ein dichtes Netz aus Vorgaben, die sich nicht mit dieser gefühlten Bequemlichkeit umgehen lassen.
Bezüglich der Frage, wie IT-Governance die Verwendung beeinflusst, lässt sich feststellen: IT-Governance wirkt als Enabler, Ermöglicherin. Durch die verpfllichtende Nutzung in professionellen Kontexten erwerben Anwender:innen notwendige technische Kompetenz, die ihnen privat oft fehlt. Dieser „Spill-over-Effekt“ führt dazu, dass berufliche Sicherheitsstandards in das Privatleben übernommen werden. Allerdings zeigt sich auch, dass schlechte Implementierung oder unbequeme Tools zu Widerstand führen, wodurch Governance kontraproduktiv auf die Nutzung wirken kann.
Für die Praxis bedeutet das schlussendlich, dass Unternehmen ihre Verantwortung neu definieren müssen. IT-Governance einerseits ein Werkzeug zur Risikominimierung in der eigenen Infrastruktur, aber auch weiters ein Instrument zur Förderung der allgemeinen digitalen Resilienz der Mitarbeitenden. Zukünftige Forschung könnte diesen Zusammenhang quantitativ prüfen, um zu messen, wie stark der Zusammenhang zwischen beruflichem Zwang zur Sicherheit und privater Sicherheitskompetenz tatsächlich ist.
