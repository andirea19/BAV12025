\section{Aufgabenstellung}
Kurzfassung
Digitale Authentifizierungstechnologien befinden sich in einem strategischen Spannungsfeld zwischen Sicherheit und Usability, welches in der vorliegenden Seminararbeit vorgestellt und diskutiert werden soll. Diese Vielschichtigkeit stellt Unternehmen und Personen vor die Herausforderung, wie sie sensible Daten und Zugriffe darauf nachhaltig sichern können. Ein-Faktor-Authentifizierung kann dieses Problem nicht (mehr) lösen, da dieser Faktor häufig ein singuläres Passwort ist, dessen Anfälligkeit für Manipulation umfangreich dokumentiert wurde (Joseph Bonneau et al., 2015, S. 81-82). 
Gleichzeitig verzeichnen mehrere europäische Organisationen einen starken Anstieg von identitätsbezogenen Angriffen gegen Einzelpersonen und Unternehmen (European Union Agency for Cybersecurity., 2024, S. 27). Multi-Faktor-Authentifizierung schließt diese Lücke, indem wie im Namen ersichtlich, unterschiedliche zusätzliche Faktoren Teil der Identitätsfeststellung werden (Lawrence O’Gorman & O’Gorman, 2003, S. 2022-2024). 
Weiters nehmen die Bedrohungen im Internet zu. Angriffsvektoren wie Phishing, Identitätsbetrug und Malware bedrohen Privatpersonen und Unternehmen gleichermaßen. Ein-Faktor-Authentifizierungssysteme stellen in diesem Zusammenhang ein großes Risiko dar, weil sie leicht umgangen werden können, und vor allem in Form von Passworten anfällig sind für menschliche Faktoren, wie Bequemlichkeit oder Vergesslichkeit (Das et al., 2020, S. 5446).
Insbesondere sensible Bereiche wie das Gesundheitswesen, Banken sowie die öffentlichen Verwaltung müssten mit großen Folgen einer Kompromittierung der Daten rechnen, weshalb Guidelines und Standards dringend notwendig sind (EBA publishes an Opinion on the elements of strong customer authentication under PSD2 | European Banking Authority, o. J., S. 12).
Dieser wissenschaftliche Text soll untersuchen, welche Faktoren auf die Verwendung von MFA-Technologien einwirken. Dabei liegt der Fokus nur teilweise auf den Bereich Usability und Akzeptanz der Technologie, sondern vorwiegend auf dem Bereich Governance als Einflussfaktor. Dieser wird vor allem in Hinblick auf die wachsende Menge an IT-Regularien und Governance-Anforderungen in unterschiedlichen Kontexten zunehmend auch für Privatpersonen beziehungsweise in privaten Kontexten relevant.  

