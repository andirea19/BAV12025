\section{Zusammenfassung}
Die vorliegende Arbeit hat gezeigt, dass die technische Sicherheit von Multifaktor-Authentifizierung allein nicht ausreicht, um eine flächendeckende Nutzung zu gewährleisten. Vielmehr hängt ihre Wirksamkeit maßgeblich von der Benutzbarkeit und der Akzeptanz der Anwender:innen ab. Wie die Literaturanalyse verdeutlichte, wird MFA von Nutzer:innen häufig als Störfaktor im Arbeitsfluss oder als zeitraubend wahrgenommen. Insbesondere im privaten Kontext, wo Nutzer:innen bei technischen Problemen auf sich allein gestellt sind, Risiken nur diffus wahrnehmen, und kein institutioneller Druck besteht, führt diese wahrgenommene Hürde oft zur Umgehung von Sicherheitsmechanismen.
In diesem Spannungsfeld tritt IT-Governance als entscheidende Treiberin auf. Im beruflichen Umfeld ersetzt sie die oft fehlende intrinsische Motivation durch regulatorische Notwendigkeit. In der Literaturbetrachtung zeigt sich, dass Governance hier eine Doppelfunktion erfüllt: Einerseits erzwingt sie die Nutzung durch Richtlinien und technische Kontrollen. Andererseits unterstützt sie diesen Zwang durch Support-Strukturen (zum Beispiel Helpdesk, Schulung). Diese senken die Hürden bei der Einrichtung und erleichtern die Fehlerbehebung, und sind im privaten Kontext meistens nicht vorhanden.
Ein zentrales Ergebnis dieser Arbeit ist die Beschreibung des sogenannten „Spill-over-Effekts“ als Bindeglied zwischen beruflichen und privaten Kontexten. Die betrachteten Forschungsdokumente legen nahe, dass die verpflichtende Nutzung in beruflichen Kontexten auch die Akzeptanz im privaten Kontext fördern kann. Indem Nutzer:innen im geschützten Unternehmensumfeld Kompetenzen aufbauen und die Verwendung von Technologie normalisieren, sinkt die Hemmschwelle, diese auch privat einzusetzen.
Die zentrale Forschungsfrage, wie IT-Governance die Verwendung von MFA beeinflusst, lässt sich somit beantworten: IT-Governance wirkt als Katalysator sowohl durch Supportmaßnahme wie zur Sensibilisierung. Indem Governance im Beruf technische Standards durchsetzt und dafür sensibilisiert, wirkt sie auch in anderen Kontexten, also über Unternehmensgrenzen hinaus und leistet indirekt einen Beitrag zur Erhöhung der privaten Sicherheitskompetenz der Mitarbeitenden.
