\documentclass[12pt,a4paper]{article}
\usepackage[utf8]{inputenc}
%ENtfernt rote ränder von Hyperlinks
\usepackage[hidelinks]{hyperref}
%Wird nicht mehr benötigt damit Umlaute im PDF richtig angezeigt werden
%\usepackage[T1]{fontenc}	

%Wird fürs Anzeigen von Grafiken benötigt
\usepackage{graphicx}

%Wird benötigt damit Tabellenbeschriftung auf Deutsch
\usepackage[ngerman]{babel} 

% verwendet für subfigure Umgebungen
\usepackage{subcaption}

% verwendet für todo notes (\todo)
\setlength {\marginparwidth }{2cm}
\usepackage[colorinlistoftodos]{todonotes}

%Package für Listing (code listing - Umgebung "lstlisting")
\usepackage{listings}

%Packages für Quellenverzeichnis
% Package für \url Befehl
\usepackage{url}
% Package-Abhängigkeit für \appto Befehl (wird für den nächsten Kommando (UrlBreaks) benötigt)
\usepackage{etoolbox}

%wird für autoref benötigt, andererseits erstellt es im pdf klickbare Referenzen
\usepackage{hyperref}
\usepackage{amsmath}
%wird für cref benötigt
\usepackage[ngerman]{cleveref}

\usepackage{relsize}
\usepackage[ngerman]{babel}
\usepackage{gensymb} 
\usepackage{textcomp}

\usepackage[sfdefault]{arimo}
% Font encoding 
\usepackage[T1]{fontenc}  
% This package allows the user to specify the input encoding 
\usepackage[utf8]{inputenc}
\usepackage{setspace}

%eventuell benötigte Pakete

 % Das Paket wrapfig ermöglicht es von Schrift umflossene Bilder und Tabellen
% \usepackage{wrapfig}

% ermöglicht das anpassen der list Umgebungen (itemize, enumerate, description)
% \usepackage{enumitem}

% bessere Möglichkeiten in die Platzierung von Abbildungen, etc. einzugreifen
% \usepackage{float}

% Abstand zwischen zwei Absätzen kontrollieren 
% \usepackage{parskip}

% konfiguriert Parameter des hyperref packages, so dass Linkfarben angepasst werden - muss die parameterlose Variante des Befehls (\usepackage{hyperref}) ersetzen
% \usepackage[colorlinks=true, urlcolor=blue, linkcolor=black]{hyperref}

\usepackage[acronym]{glossaries}
\usepackage{blindtext}
\usepackage{tocloft}
\usepackage{lastpage}

%\do\? == erlaubtes Zeichen, an dem ein Zeilenumbruch innerhalb der URL stattfinden darf (auch mehrere \do\? erlaubt, z.B. \do\a\do\b\do\c...)
\appto\UrlBreaks{\do\4}
\makeglossaries


%Metadaten
% werden verwendet, wenn z.B. für das Titelblatt/Deckblatt der Befehl \maketitle verwendet wird
\title{FH-Burgenland Vorlage V2}
\author{Martin Scheifinger}
\date{\today} %{11.10.2020}


%Commandodefinition
% Defines a new command for the horizontal lines, change thickness here
% \newcommand{\HRule}{\rule{\linewidth}{0.5mm}} 

%Seitenränder und Druckbereich sowie Kopf- und Fußzeile definieren
\setlength\paperwidth{20.999cm}
\setlength\paperheight{29.699cm}
\setlength\voffset{-2cm}
\setlength\hoffset{-1cm}
\setlength\headheight{3.5cm}
\setlength\headsep{0.5cm}
\setlength\footskip{1.131cm}
\setlength\textheight{22cm}
\setlength\textwidth{17cm}
%\setlength\marginparwidth{2cm}

%%%%%%%%%%%%%%%%%%%%%%%%%%%%%%%%%%%%%%%%%%%% Kopfzeile und Fußzeile definieren

\usepackage{fancyhdr}
\pagestyle{fancy} %eigener Seitenstil
\fancyhf{} %alle Kopf und Fußzeilenfelder bereinigen
\fancyhead[L]{\KlassenNr - \FachKuerzel}
\fancyhead[R]{\includegraphics[height=1.7cm]{FH-Burgenland_Logos/Hochschule_Bgld_Logo_RGB.png}}


\let\oldheadrule\headrule% Copy \headrule into \oldheadrule

%uncommend below to make header line türkis
%\renewcommand{\headrule}{\color{teal}\oldheadrule}% Add colour to \headrule

\renewcommand{\headrulewidth}{0.5pt}%obere Trennlinie
%\renewcommand{\footrulewidth}{0.0pt} %untere Trennlinie definieren
\fancyfoot[L]{\SchuelerName}
\fancyfoot[R]{Seite \thepage \hspace{1pt} von \pageref{LastPage}}
\pagenumbering{arabic}

%%%%%%%%%%%%%%%%%%%%%%%%% Inhaltsverzeichnis Formatierung definieren

\cftsetindents{section}{0em}{3em}
\cftsetindents{subsection}{0em}{3em}
\cftsetindents{subsubsection}{0em}{3em}

\setcounter{tocdepth}{4} %Damit Inhaltsverzeichnis bis zur 4 Ebene
\renewcommand\thesection{\arabic{section}} %Damit die Nummerierung mit 1 beginnt, vgl.: \roman{section}

\setcounter{secnumdepth}{3} %Damit Nummerierung bis zur Ebene 3 geht






%%%%%%%%%%%%%%%%%%%%%%%%%%%%%%%%%%%%%%%%%%%%%%%%%%%%%%%%%%%%%%%%%%%%%%%%%%%%%%%%
%Start vom eigentlichen Dokument / Schriftstück 
\begin{document}
%Deckblatt
%--------------Wichtige Metadaten für jedes Dokument ------------------
\def\FachKuerzel{FACH}
\def\KlassenNr{MCCE - 1} %z.B. BITI - 5V
\def\Titel{--TITEL--}
\def\Ueberschrifta{--Überschrift 1 --}
\def\Ueberschriftb{--Überschrift 2 --} %Falls nicht benötigt in File Dekblatt Zeile 19 auskommentieren
\def\Lehrer{--Lehrer--}


%--------------- Name des Schülers ------------------------
\def\SchuelerName{-- Name --}
\def\SchuelerPKZ{-- PKZ --}


%--------------- Department Settings ------------------------
\def\Department{--Department--} % z.B. Informationstechnologien
\def\Studiengang{--Studiengang--} % z.B. IT-Infrastruktur Management

%------------------------------------------------------------


\begin{titlepage}
\begin{center}
    % UCL IMAGE
    \vspace*{-5,7cm}
    \hspace*{-1,43cm}
    \makebox[\textwidth]{\includegraphics[width=\paperwidth]{FH-Burgenland_Logos/image1.png}}
    
    \vspace{2.3cm}
    

    \vspace{1cm}
    {\relscale{2.00}\textbf{\Titel\\}}
    \vspace{2.4cm}
    
    {\relscale{2.00}{\Ueberschrifta}} \\
    \vspace{0.5cm}
    {\relscale{2.00}{\Ueberschriftb}}\\
    \vspace{0.5cm}
    Von\\
    %\vspace{0.1cm}
    \SchuelerName \\ \SchuelerPKZ\\
    \vspace{0.9cm}
    {\begin{singlespace}Kurs gehalten von:\\\end{singlespace}}
    {\begin{singlespace}\Lehrer\\\end{singlespace}}


\end{center}
{\raggedleft\vfill{\begin{singlespace}
     Department \Department :\\
\end{singlespace}
 \Studiengang\\
 \begin{singlespace}
 Datum: \today\\
 
 
 \end{singlespace}
}\par
}
\end{titlepage}


\newpage
%%%%%%%%%%%%%%%%%%%%%%%%%%%%%%%%%%%% Ende Deckblatt
%%%%%%%%%%%%%%%%%%%%%%%%%%%%%%%%%%%% Aktivieren der verschiedenen Verzeichnisse
% wobei Quell- und Literaturverzeichnis sich am Ende des Dokumentes befinden.

\tableofcontents\thispagestyle{fancy}


\pagenumbering{arabic}
%\setcounter{page}{1}

%%%%%%%%%%%%%%%%%%%%%%%%%%%%%%% Ende Verzeichnis
%_______________________________________________________________________________________________________
%   ###########################################################
%   #           Metadaten ändern nicht vergessen              #
%   ###########################################################
% Dokument: main
%   Zeile: 36    | Header: Fach Hinzufügen ()
% Dokument: Deckblatt:
%   Zeile: 12    | Titel des Dokuments     ()
%   Zeile: 15&17 | Überschriften           ()
%   Zeile: 24    | Vortragende Person      ()
%_______________________________________________________________________________________________________
%--------------Wichtige Metadaten für jedes Dokument ------------------
\def\FachKuerzel{FACH}
\def\KlassenNr{MCCE - 1} %z.B. BITI - 5V
\def\Titel{--TITEL--}
\def\Ueberschrifta{--Überschrift 1 --}
\def\Ueberschriftb{--Überschrift 2 --} %Falls nicht benötigt in File Dekblatt Zeile 19 auskommentieren
\def\Lehrer{--Lehrer--}


%--------------- Name des Schülers ------------------------
\def\SchuelerName{-- Name --}
\def\SchuelerPKZ{-- PKZ --}


%--------------- Department Settings ------------------------
\def\Department{--Department--} % z.B. Informationstechnologien
\def\Studiengang{--Studiengang--} % z.B. IT-Infrastruktur Management

%------------------------------------------------------------

\clearpage



%%%%%%%%%%%%%%%%%%%%%%%%%%%%%%%%%%%%% Anfang Dokument %%%%%%%%%%%%%%%%%%%%%%%%%%%%%%%%%%%%%%%%%%%%%%

% ----------------------  Kapitel 1 -----------------------
\newpage

\section{Aufgabenstellung}
Kurzfassung
Digitale Authentifizierungstechnologien befinden sich in einem strategischen Spannungsfeld zwischen Sicherheit und Usability, welches in der vorliegenden Seminararbeit vorgestellt und diskutiert werden soll. Diese Vielschichtigkeit stellt Unternehmen und Personen vor die Herausforderung, wie sie sensible Daten und Zugriffe darauf nachhaltig sichern können. Ein-Faktor-Authentifizierung kann dieses Problem nicht (mehr) lösen, da dieser Faktor häufig ein singuläres Passwort ist, dessen Anfälligkeit für Manipulation umfangreich dokumentiert wurde (Joseph Bonneau et al., 2015, S. 81-82). 
Gleichzeitig verzeichnen mehrere europäische Organisationen einen starken Anstieg von identitätsbezogenen Angriffen gegen Einzelpersonen und Unternehmen (European Union Agency for Cybersecurity., 2024, S. 27). Multi-Faktor-Authentifizierung schließt diese Lücke, indem wie im Namen ersichtlich, unterschiedliche zusätzliche Faktoren Teil der Identitätsfeststellung werden (Lawrence O’Gorman & O’Gorman, 2003, S. 2022-2024). 
Weiters nehmen die Bedrohungen im Internet zu. Angriffsvektoren wie Phishing, Identitätsbetrug und Malware bedrohen Privatpersonen und Unternehmen gleichermaßen. Ein-Faktor-Authentifizierungssysteme stellen in diesem Zusammenhang ein großes Risiko dar, weil sie leicht umgangen werden können, und vor allem in Form von Passworten anfällig sind für menschliche Faktoren, wie Bequemlichkeit oder Vergesslichkeit (Das et al., 2020, S. 5446).
Insbesondere sensible Bereiche wie das Gesundheitswesen, Banken sowie die öffentlichen Verwaltung müssten mit großen Folgen einer Kompromittierung der Daten rechnen, weshalb Guidelines und Standards dringend notwendig sind (EBA publishes an Opinion on the elements of strong customer authentication under PSD2 | European Banking Authority, o. J., S. 12).
Dieser wissenschaftliche Text soll untersuchen, welche Faktoren auf die Verwendung von MFA-Technologien einwirken. Dabei liegt der Fokus nur teilweise auf den Bereich Usability und Akzeptanz der Technologie, sondern vorwiegend auf dem Bereich Governance als Einflussfaktor. Dieser wird vor allem in Hinblick auf die wachsende Menge an IT-Regularien und Governance-Anforderungen in unterschiedlichen Kontexten zunehmend auch für Privatpersonen beziehungsweise in privaten Kontexten relevant.  


% the Asterix (*) indicates that this section will be added to the table of contents but no number will be present beside it.
\section{Einleitung}
\subsection{Problemstellung}
Digitale Authentifizierungstechnologien bewegen sich in einem strategischen Spannungsfeld zwischen Sicherheit und Usability. Diese Dualität stellt Unternehmen wie Privatpersonen vor die Herausforderung, sensible Daten und Zugriffe nachhaltig zu sichern. Die klassische Ein-Faktor-Authentifizierung (1FA) kann dieses Problem nicht mehr lösen, da der Faktor „Wissen“ (meist ein Passwort) durch menschliche Schwächen wie Vergesslichkeit oder Bequemlichkeit anfällig ist und dessen Manipulierbarkeit in der Literatur umfangreich dokumentiert wurde (Bonneau et al., 2015, S. 81).
Gleichzeitig verzeichnen europäische Organisationen einen massiven Anstieg von identitätsbezogenen Angriffen (European Union Agency for Cybersecurity, 2024, S. 27). Angriffsvektoren wie Phishing, Identitätsbetrug und Malware bedrohen dabei private und korporative Akteure gleichermaßen. In diesem Kontext stellt die alleinige Nutzung von Passwörtern ein inakzeptables Risiko dar (Das et al., 2020, S. 5446).
Multifaktor-Authentifizierung (MFA) schließt diese Sicherheitslücke, indem sie zusätzliche, (voneinander) unabhängige Faktoren in die Identitätsfeststellung integriert (O’Gorman, 2003, S. 2023). Insbesondere in sensiblen Sektoren wie dem Gesundheitswesen, dem Bankensektor oder der öffentlichen Verwaltung hätte eine Kompromittierung verheerende Folgen, weshalb verbindliche Standards und Guidelines hier dringend erforderlich sind (EBA, o. J., S. 12).
Die vorliegende Arbeit untersucht, welche Faktoren die effektive Nutzung von MFA beeinflussen. Während Aspekte der Usability und Akzeptanz als Rahmenbedingungen betrachtet werden, liegt der primäre Fokus auf IT-Governance als steuerndem Element. Obwohl Governance üblicherweise im Unternehmenskontext verortet wird, gewinnt sie durch die wachsende Anzahl an IT-Regularien (NIS-2 , DORA, FIDA, CRA) zunehmend auch für Privatpersonen an Relevanz, da berufliche Sicherheitsvorgaben und private Nutzungsgewohnheiten zunehmend verschwimmen.
\subsection{Zielsetzung und wissenschaftliche Fragestellung}
Multifaktor-Authentifizierung (MFA) stellt weit mehr dar als eine rein technische Standardlösung für das Identitäts- und Zugangsmanagement. Sie bewegt sich in einem komplexen Anforderungsfeld: Die Technologie muss nicht nur kryptografisch sicher und resistent gegen Täuschung sein, sondern gleichzeitig externe Compliance-Anforderungen erfüllen und durch eine hohe Benutzerfreundlichkeit (Usability) Risiken minimieren, die durch menschliches Fehlverhalten entstehen (Kim et al., 2011, S. 187).
In diesem Kontext nimmt IT-Governance eine Schlüsselrolle ein. Durch Richtlinien und strategische Vorgaben können Unternehmen Sicherheitsaspekte mit den Interessen der Stakeholder in Einklang bringen (Wibowo & Ramli, 2022, S. 2). Auf Basis der ermittelten Authenticator Assurance Levels (AAL) werden Technologien und Prozesse definiert, um diese Ziele zu erreichen. Damit verortet sich das Thema nachvollziehbar im Bereich des strategischen IT-Infrastruktur-Managements.
Die zentrale Forschungsfrage der vorliegenden Arbeit lautet:
Inwiefern wirkt IT-Governance als Treiber für die Akzeptanz und Nutzung von Multifaktor-Authentifizierung im Vergleich zwischen beruflichen und privaten Kontexten?
Zur detaillierten Beantwortung werden folgende Unterfragen untersucht, um die regulatorischen und psychologischen Dimensionen herauszuarbeiten:
Welche regulatorischen und organisationalen Mechanismen erzwingen die Nutzung im beruflichen Umfeld?
Welche Barrieren verhindern die freiwillige Nutzung im privaten Umfeld?
Methodisch erfolgt die Bearbeitung durch eine systematische Literaturrecherche (Literature Review). Dabei werden relevante wissenschaftliche Publikationen, Fachartikel und Standards (z. B. NIST, ISO) analysiert, um sowohl die technischen Aspekte von MFA als auch die organisatorischen Governance-Faktoren zu beleuchten. Ziel ist es, die Nutzungsmuster im beruflichen und privaten Kontext gegenüberzustellen, Unterschiede herauszuarbeiten sowie Erfolgsfaktoren und Hindernisse zu identifizieren.
Die Relevanz dieser Seminararbeit ergibt sich insbesondere für (angehende) IT-Manager:innen sowie Verantwortliche im Bereich Identity- und Access-Management (IAM), um Entscheidungen mit stärkerem Nutzer:innen-Fokus zu treffen und Organisationen auf die steigende Dichte an regulatorischen Vorgaben vorzubereiten.



\input{Kapitel/6_Fazit.tex}


%%%%%%%%%%%%%%%%%%%%%%%%%%%%%%%%%%%%%%%%%%%%%%%%%%%%%%%%%%%%%%%%%%%%%%%%%%%%%%%%%%%%%%%%%%%%%%%%%%%%
\newpage
\section{Anhang}
%Hier einkommentieren wenn man Abbildungsverzeichnis oder Tabellenverzeichnis macht
%\listoffigures\thispagestyle{fancy}
%\clearpage
%\listoftables\thispagestyle{fancy}
%\clearpage
% -----------------  ACRONYMS  -------------------


\printglossary[type=\acronymtype,nonumberlist,nopostdot]



\end{document}
